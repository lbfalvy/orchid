\documentclass{article}
\usepackage{graphicx}
\usepackage[margin=2cm]{geometry}
\usepackage[hidelinks]{hyperref}


\title{Orchid's Type System}
\author{Lawrence Bethlenfalvy, 6621227}
\date{12 November 2022}

% Why would you toss all this in the template if it just doesn't compile!?
%\urn{6621227}
%\degree{Bachelor of Science in Computer Science}
%\supervisor{Brijesh Dongol}

\begin{document}
\maketitle


\section{Introduction}

Originally my final year project was going to be an entire programming language which I started to
develop around February, however at the start of the year I decided to set a more reasonable goal.

Orchid is a functional programming language inspired by $\lambda$-calculus, Haskell and Rust. The
execution model is exactly $\lambda$-calculus with opaque predicates and functions representing
foreign data such as numbers, file descriptors and their respeective operations. For the purpose of
side effects caused by foreign functions, reduction is carried out in normal order just like
Haskell.

There are two metaprogramming systems, one syntax level and one type level, similar to Rust.
Syntax-level metaprogramming is based on generalized kerning, it is mostly defined and a naiive
implementation is complete at the time of writing. Type-level metaprogramming resembles Prolog and
is a major objective of this year's project.

The project's home is this repository which, at the time of writing, contains fairly outdated code
samples: \url{https://github.com/lbfalvy/orchid}

\subsection{Aims}

My goal for this year is to define a robust, usable type system and write a performant
implementation.

The next phase of development will be a compiler frontend for LLVM. If the type system reaches a
satisfactory level of completion before the dissertation is complete, I may also write a bit about
the compilation.

If due to some unforeseen circumstances I'm unable to complete enough of the type system to fill
the dissertation with its details or it ends up too simple, I may also write about the macro system
which is already in a usable state and only needs some optimizations and minor adjustments
due to shifts in responsibilities which occured while I was defining the basic properties of the
type system and experimenting with concrete code examples to get a clearer picture of the
provisional feature set.

\subsection{Objectives}

A working type system should have the following parts, which I will implement in roughly this order

\begin{itemize}
    \item \textbf{Type inference engine and type checker} This will be an extension of
        the Hindley-Milner algorithm, which simultaneously unifies and completes partial type
        annotations, and recognizes conflicts.
    \item \textbf{Typeclass solver} At the moment this appears to be a relatively simple piece of
        code but I'm not entirely confident that complications won't arise as its responsibilities
        become clearer, so I consider it a separate component
    \item \textbf{Executor} Orchid is a statically typed language so it should eventually be compiled
        with LLVM, but in order to demonstrate the usability of my type system I will have to write
        an experimental interpreter. Since types are already basically expressions of type type,
        parts of the executor will coincide with parts of the type inference engine.
\end{itemize}

\section{Literature Review}

The preprocessor can parse arbitrary syntax. Generalized kerning can use "rockets"
(called carriages in Orchid terminology) to parse token sequences statefully and assume
the role of an arbitrary parser encoded as a rich Turing machine.\cite{suckerpinch}

The type system supports higher-kinded types. I considered translating higher-kinded polymorphism
into abstract types as demonstrated by Yallop\cite{yallop} which can be translated into
Prolog and then building the breadth-first executor described by Tubella\cite{tubella}, but
in the end I decided that since I'm already building my own unification system I might as well
skip this step. Currently expressions are annotated with common Orchid expressions that evaluate to
a type. This means that unification is uncomputable in some cases, but the most common cases
such as halting expressions and recursive types using fixed point combinators can be unified
fairly easily and this leaves room for extension of the unification engine.

\section{Technical Overview}

\subsection{Type checker}

Type expressions to be unified are collected into a group. For the purpose of unification, types
are either opaque types with possible parameters which are considered equal if both the type and its
parameters are equal, or transparent lambda expressions applied to types. Before unification would
begin, the expressions that refer to equal types are collected in a group. A breadth-first search
through the network of reduced forms is executed for all expressions in lockstep, and
syntactic unification is attempted on each pair of reduced forms belonging to different expressions
in the same group.

At a minimum, the following must be valid reduction steps:

\begin{itemize}
    \item $\beta$-reduction
    \item fixed point normalization, which simply means identifying that a subexpression has
        reduced to an expression that contains the original. When a fixed point is detected, the
        recursive expression is converted to a form that uses the Y-combinator. This operation
        is ordered before $\beta$-reductions of the expression in the BFS tree but otherwise has
        the same precedence.
\end{itemize}

\subsection{Typeclass solver}

This will be relatively simple and strongly resemble Rust's Chalk trait solver, with the exception
that I would prefer not to enforce the orphan rules on the language level so as not to completely
stall projects while a third party developer accepts pull requests on what might be legacy code to
add new impls.

\subsection{Executor}

A basic version of the executor can technically be produced by initializing the lazy BFS of
reductions created for the type checker on runtime code, taking the first result, dropping the
BFS iterator and repeating these two steps ad infinitum, but this will likely be very inefficient
so time permitting I would like to create a somewhat better version. This stands to show how
developer effort can be reduced - along with the learning curve of the complex type system - by
reusing the same language for both. A type system supporting HKTs would have to be uncomputable
either way.

\section{Workplan}

TODO

\appendix

\bibliographystyle{IEEEtran}
\bibliography{references}

\end{document}